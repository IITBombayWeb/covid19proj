\documentclass{article}
\usepackage{amsmath}
\usepackage{bm}
\usepackage{booktabs}
%\usepackage{longtable}
\usepackage{tabularx}
%\usepackage{ltablex}
\usepackage[super]{natbib}
\usepackage{graphicx}
\graphicspath{{images/}}
\usepackage{fullpage}
\usepackage{pdflscape}
\usepackage{tabularx}


\usepackage{caption,subcaption}


%\usepackage[varg]{txfonts}
\usepackage[varg]{pxfonts}


\usepackage{hyperref}
\hypersetup{
    colorlinks=true,
    citecolor=black,
    filecolor=black,
    linkcolor=blue,
    urlcolor=blue,
    linktoc=all
}

\newcommand{\email}[1]{\href{mailto:#1}{#1}}

\usepackage{fancyhdr}

% Heading format
\fancyhead{} % clear old format
\fancyhead[LE,RO]{\thepage}
\if@twoside
  \fancyhead[LO]{\em\nouppercase\rightmark}
  \fancyhead[RE]{\em\nouppercase\leftmark }
\else
  \fancyhead[L]{\em\nouppercase\rightmark}
\fi

\cfoot{}


\newcommand{\nd}{\ensuremath{N_{\mathrm{d}}}}
\newcommand{\ndl}{\ensuremath{N_{\mathrm{d,low}}}}
\newcommand{\ndh}{\ensuremath{N_{\mathrm{d,high}}}}
\newcommand{\ncr}{\ensuremath{N_{\mathrm{cr}}}}
\newcommand{\nca}{\ensuremath{N_{\mathrm{ca}}}}
\newcommand{\ns}{\ensuremath{N_{\mathrm{s}}}}
\newcommand{\nin}{\ensuremath{N_{\mathrm{i}}}}
\newcommand{\nq}{\ensuremath{N_{\mathrm{q}}}}

\newcommand{\fd}{\ensuremath{f_{\mathrm{d}}}}

\newcommand{\nicu}{\ensuremath{N_{\mathrm{ICU}}}}
\newcommand{\nacu}{\ensuremath{N_{\mathrm{ACU}}}}
\newcommand{\nscu}{\ensuremath{N_{\mathrm{SCU}}}}

\newcommand{\ma}{\ensuremath{m_{\mathrm{a}}}}
\newcommand{\mi}{\ensuremath{m_{\mathrm{i}}}}

\begin{document}
\title{COVID-19 Medical Inventory Prediction}
\date{\today}
\author{
  P Sunthar\thanks{Professor, Department of Chemical Engineering, IIT
    Bombay, Mumbai \email{p.sunthar@iitb.ac.in}}
}
\maketitle
\begin{abstract}
  This document describes the assumptions behind the estimated medical
  inventory requirement for attending to COVID-19 patients in
  hospitals, as projected in \url{https://covid19medinventory.in}. The
  estimated patient counts have been obtained from
  \url{https://mesoscalelab.github.io/covid19/}. The projections are
  themselves dynamic, and can change every week depending on newly
  available evidence. The medical inventory list and the formule for
  its dependence on the type of patients is also dynamic, and will
  change as we find better relationships with actual field data.
\end{abstract}

\section{Basic Assumptions}

The number of COVID-19 positive patients obtained from the
district-wise projection\cite{ansualok20} is denoted as $N$. This could
be the current number or the projected number in future weeks.
According to WHO statistics\cite{who19mar}, the number of patients who
actually need various levels of care \cite{aiims2020a} is given in the
table below:

\noindent
\begin{tabularx}{\linewidth}{XXrc}
  \toprule
  Type of patient & Type of care & approx \% & Symbol used here \\
  \midrule
  Total positive & -- & 100\% & $N$ \\
  Mild & Symptomatic, Home Quarantine/isolation (out-patients) & 40\% & \nq \\
  Moderate & In-patient ward & 40\% & \nin \\
  Severe & Supportive care, oxygen therapy   & 15\% & \ns \\
  Critical & ICU, mechanical ventilation & 5\% & \ncr \\
  Deceased & --  & 2.5\% & \nd \\
\bottomrule  
\end{tabularx}


\section{Inventory Estimates}
The tables below gives the inventory list, the estimation logic, and
formula, and an example calculation for $N=100$ positive cases.

In the example case the patient estimates are given below. In the
website, some of the numbers are rounded higher to the nearest 10 or
50, depending on their magnitude.

\noindent
\begin{tabular}{lll}
  \toprule
  Category & Symbol & Estimate \\
  \midrule
  Total positives & $N$ & 100 \\
  Mild & $\nq$ & 40 \\
  Moderate & $\nin$ & 40 \\
  Severe & $\ns$ & 15 \\
  Critical & $\ncr$ & 5\\
  Deceased & $\nd$ & 2.5\\
  \bottomrule
\end{tabular}

ICU patients at a given day $t$
\begin{equation}
  \label{eq:nicu0}
  \begin{split}
  \nicu(t)
  &= \text{New admissions on $t$}\\
  & \quad - \text{Deaths on $t$}\\
  & \quad - \text{Discharged on $t$ after $m$ days of admission}\\
  & \quad + \text{Continuing on $t$ from the last $m-1$ days} \\
   &= \Delta \ncr (t)
    - \Delta \nd(t) 
    - (1-\fd) \, \Delta \ncr(t-m)
    + (1-\fd) \, \sum_{r=1}^{m-1} \, \Delta \ncr(t-r)
  \end{split}
\end{equation}
where, the new ICU admissions, or deaths on a day is the increase in
the number of critical patients from the previous day:
\begin{align}
  \Delta \ncr(t)  &= \ncr(t) - \ncr(t-1) \\
  \Delta \nd(t)  &= \nd(t) - \nd(t-1)
\end{align}
The number of deaths on a given day is a fraction of the admissions
$n$ days earlier:
\begin{equation}
  \Delta \nd(t) = \fd \, \Delta \ncr(t-n)
\end{equation}
or alternatively:
\begin{equation}
  \Delta \ncr(t) = \frac{1}{\fd} \, \Delta \nd(t+n)
\end{equation}
Since the model estimates the deaths correctly Eq.~\eqref{eq:nicu0} can be
rewritten as:
\begin{equation}
  \label{eq:nicu}
  \nicu(t)
  = \frac{1}{\fd} \left( \Delta \nd (t+n)
                       - \fd \Delta \nd(t) 
                       - (1-\fd) \, \Delta \nd(t-m+n)
                       + (1-\fd) \, \sum_{r=1}^{m-1} \, \Delta \nd(t-r+n)
                       \right)
\end{equation}

Similarly, we can estimate the number of persons in acute care (ACU,
for the severe cases) and supportive care (SCU for moderate and/or
mild cases).  Assuming the residence time of Acute Care is $\ma$ and
that for Supportive Care (in-patient) is $\mi$ current number of patients in each
of these care units is:
\begin{align}
  \label{eq:nacu0}
  \begin{split}
  \nacu(t)
  &= \text{New admissions on $t$}\\
  & \quad - \text{Discharged on $t$ after $\ma$ days of admission}\\
  & \quad + \text{Continuing on $t$ from the last $\ma-1$ days} \\
   &= \Delta \ns (t)
    - \Delta \ns(t-\ma)
    + \sum_{r=1}^{\ma-1} \, \Delta \ns(t-r)
  \end{split}
\end{align}
For ease The severe care patients are estimated directly from the reported
cases and not from the 


In the case of SCUs, in addition to the above, the
patients discharged from the ICUs are recuperated for the same $\mi$
days. However, this number is small compared to the number admitted
afresh (3 against 40 to 80), therefore is ignored.
\begin{align}
  \label{eq:nscu0}
  \begin{split}
  \nscu(t)
  &= \text{New admissions on $t$}\\
  & \quad - \text{Discharged on $t$ after $\mi$ days of admission}\\
  & \quad + \text{Continuing on $t$ from the last $\mi-1$ days} \\
   &= \Delta \nin (t)
    - \Delta \nin(t-\mi)
    + \sum_{r=1}^{\mi-1} \, \Delta \nin(t-r)
  \end{split}
\end{align}



%  &= \text{Critical care patients discharged from ICU}

\section{Acknowledgements}
The author thanks Dr Santosh Ansumali (JNCASR) and Dr. Aloke Kumar
(IISc) for providing the projection maps, and Dr. Soumyadeep
Bhattacharya (Sankhyasutra Labs Pvt. Ltd) for data API. We also thank
the office of the Principal Scientific Advisor of Govt. of India for
providing an elaborate list of inventory from which some critical ones
have been used here.

\section{Errors and Omissions}
Though the authors have taken sufficient care in estimating the
quantity of the inventory, the proportions used here may change
depending on the actual field data in India for the patients and the
inventory usage pattern.  Please email your comments and suggestions
with justifications to \email{p.sunthar@iitb.ac.in} for inclusion in the
future updates of the software.

\appendix
\section{Patient projection model}
The key formulae from patient prediction model \cite{ansualok20} is
given here. Let the case fatality rate (CFR) as defined by number
deceased \nd\ to
the total number of cases $N$ be
\begin{equation}
  m = \frac{\nd}{N}
\end{equation}
It is assumed that the CFR is same for all districts of a state.


The number of critical patients \ncr\ is taken to be bounded by the
two different death rates (as per predictions given under):
\begin{equation}
  2 \, \ndl \leq \ncr \leq 4 \, \ndh
\end{equation}
where the lower bound factor (of 2) is based on world averages\cite{who19mar} and
the upper bound is based on ??. Here, \ndl\ and \ndh\ are the deaths predicted by
the lower and higher death-growth rates, respectively (see below). 

\subsection{Growth predictions}
The weekly growth of number of deaths is given by
\begin{equation}
  \label{eq:gnd}
  \nd(t+n) = g(n) \, \nd(t)  = g(n) \, m \, N(t)
\end{equation}
where $g$ is a weekly growth factor and $n$ is the number of weeks
from a reference date at $t$.  There are two scenarios for the values
of $g$: Low growth and High growth, which are bounds of the death
growth rates obtained by fitting the growth rates across the
world\cite{ansualok20}.  The low growth
factor for values of $n = \{1, 2, 3, 4\}$:
\begin{equation}
  g_{\mathrm{low}} = \begin{cases}
   \{4, 27.85, 85, 160\}  & \nd < 10 \\
   \{5.5, 20. 70, 150\} & \nd \geq 10
 \end{cases}
\end{equation}
Similarly, the high growth factor for values of $n = \{1, 2, 3, 4\}$:
\begin{equation}
  g_{\mathrm{high}} = \begin{cases}
   \{6.5, 42, 110, 300\}  & \nd < 10 \\
   \{7.6, 35, 108, 230\} & \nd \geq 10
 \end{cases}
\end{equation}
These growth factors provide the corresponding projected $\ndl$ and $\ndh$ from
Eq.~\eqref{eq:gnd}.



\bibliography{covid19lit}
\bibliographystyle{unsrtnat}

\end{document}
%%% Local Variables:
%%% mode: latex
%%% TeX-master: t
%%% End:
