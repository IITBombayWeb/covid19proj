\documentclass{article}
\usepackage{amsmath}
\usepackage{bm}
\usepackage{booktabs}
%\usepackage{longtable}
\usepackage{tabularx}
%\usepackage{ltablex}
\usepackage[super]{natbib}
\usepackage{graphicx}
\graphicspath{{images/}}
\usepackage{fullpage}
\usepackage{pdflscape}
\usepackage{tabularx}


\usepackage{caption,subcaption}


%\usepackage[varg]{txfonts}
\usepackage[varg]{pxfonts}


\usepackage{hyperref}
\hypersetup{
    colorlinks=true,
    citecolor=black,
    filecolor=black,
    linkcolor=blue,
    urlcolor=blue,
    linktoc=all
}

\newcommand{\email}[1]{\href{mailto:#1}{#1}}

\usepackage{fancyhdr}

% Heading format
\fancyhead{} % clear old format
\fancyhead[LE,RO]{\thepage}
\if@twoside
  \fancyhead[LO]{\em\nouppercase\rightmark}
  \fancyhead[RE]{\em\nouppercase\leftmark }
\else
  \fancyhead[L]{\em\nouppercase\rightmark}
\fi

\cfoot{}


\newcommand{\nd}{\ensuremath{N_{\mathrm{d}}}}
\newcommand{\nc}{\ensuremath{N_{\mathrm{c}}}}
\newcommand{\ns}{\ensuremath{N_{\mathrm{s}}}}
\newcommand{\nin}{\ensuremath{N_{\mathrm{i}}}}
\newcommand{\nq}{\ensuremath{N_{\mathrm{q}}}}




\begin{document}
\title{COVID-19 Medical Inventory Prediction}
\date{\today}
\author{Deepak Padmanabhan\thanks{MBBS, MD DM, Cardiologist, Sri Jayadeva
    Institute of Cardiology, Bengaluru}
  \and
  P Sunthar\thanks{Professor, Department of Chemical Engineering, IIT
    Bombay, Mumbai \email{p.sunthar@iitb.ac.in}}
}
\maketitle
\begin{abstract}
  This document describes the assumptions behind the estimated medical
  inventory requirement for attending to COVID-19 patients in
  hospitals, as projected in \url{https://covid19medinventory.in}. The
  estimated patient counts have been obtained from
  \url{https://mesoscalelab.github.io/covid19/}. The projections are
  themselves dynamic, and can change every week depending on newly
  available evidence. The medical inventory list and the formule for
  its dependence on the type of patients is also dynamic, and will
  change as we find better relationships with actual field data.
\end{abstract}

\section{Basic Assumptions}

The number of COVID-19 positive patients obtained from the
district-wise projection\cite{ansualok20} is denoted as $N$. This could
be the current number or the projected number in future weeks.
According to WHO statistics\cite{who19mar}, the number of patients who
actually need various levels of care \cite{aiims2020a} is given in the
table below:

\noindent
\begin{tabularx}{\linewidth}{XXrc}
  \toprule
  Type of patient & Type of care & approx \% & Symbol used here \\
  \midrule
  Total positive & -- & 100\% & $N$ \\
  Mild & Symptomatic, Home Quarantine/isolation (out-patients) & 40\% & \nq \\
  Moderate & In-patient ward & 40\% & \nin \\
  Severe & Supportive care, oxygen therapy   & 15\% & \ns \\
  Critical & ICU, mechanical ventilation & 5\% & \nc \\
  Deaths & --  & 2.5\% & \nd \\
\bottomrule  
\end{tabularx}


\section{Inventory Estimates}
The tables below gives the inventory list, the estimation logic, and
formula, and an example calculation for $N=100$ positive cases.

In the example case the patient estimates are given below. In the
website, some of the numbers are rounded higher to the nearest 10 or
50, depending on their magnitude.

\noindent
\begin{tabular}{lll}
  \toprule
  Category & Symbol & Estimate \\
  \midrule
  Total positives & $N$ & 100 \\
  Mild & $\nq$ & 40 \\
  Moderate & $\nin$ & 40 \\
  Severe & $\ns$ & 15 \\
  Critical & $\nc$ & 5\\
  Deaths & $\nd$ & 2.5\\
  \bottomrule
\end{tabular}


\begin{landscape}
\begin{table}
  \caption{Persons and Protective Equipment. The last column is
    rounded number assuming a total positive cases of $N=100$.}
  \begin{tabularx}{\linewidth}{XXcl}
    \toprule
    Item [Unit] & Reason & Formula & $N=100$ \\
    \midrule
Doctors [per day] & 2 per 12-hr shift for 10 critical and severe
patients and 2 
per 12-hr shift for 15 in-patients &
$\frac{2}{5}\left(\nc+\ns+\frac{\nin}{1.5}\right)$ & 19 \\
Nurses [per day] & 1 per 8-hr shift for 3 critical and severe patients
and 1 per 8-hr shift for 5 in-patients &
$\left(\nc+\ns+\frac{\nin}{1.5}\right)$ & 47\\
% Janitors [per day] & 1 per 8-hr shift per 10 beds &
% $3 \; \left(\nc+\ns+\nin\right)/10$ \\
Total staff [per day]& Doctors, Nurses &
$S \equiv \frac{7}{5} \left(\nc+\ns+\frac{\nin}{1.5}\right)$ & 66 \\
Staff PPE (Gowns, Masks, Goggles etc.) [per day]&
Two per staff & $2 \, S$  & 132\\
Patient PPE: Masks [per day] & 4 per severe and moderate patients &
$4 (\ns+\nin)$ & 220 \\
Sterile gloves [per day]&
 3 per patient per 6-hr shift for critical and half that for severe and
 moderate &
$12\nc + 6 (\ns+\nin)$ & 390 \\
Non-sterile gloves &
 6 per patient per 6-hr shift for critical and half that for severe and
 moderate &
$24\nc + 12 (\ns+\nin)$ & 780 \\
Dead body bags & 1 per death & $\nd$ & 3 \\
    \bottomrule
  \end{tabularx}
\end{table}
\end{landscape}


\begin{landscape}
\begin{table}
  \caption{Medical Equipment. The last column is
    rounded number assuming a total positive cases of $N=100$.}
  \begin{tabularx}{\linewidth}{XXcl}
    \toprule
    Item [Unit] & Reason & Formula & $N=100$\\
    \midrule
    Ventilators, Ambu bags, Glass case & 1 per critical patient&
    $\nc$ & 5 \\
    Laryngoscopes, Defibrillator & 3 per 10 critical patients &
    $3 \nc/10$ & 2 \\
    ECG & 1 per 20 critical patients& $\nc/20$ & 1\\
    Arterial BP monitors & 1 per critical patient &
    $\nc$ & 5 \\
    Arterial blood gas machine & 1 per 30 critical patients &
    $\nc/30$ & 1 \\
    Bedside X-ray & 1 per 20 critical patients & $\nc/20$ & 1\\
    Infusion pumps & 3 per critical patient & $3 \nc$ & 15 \\
    Oxymeter &  1 per 20 severe patients & $\ns/20$ & 1\\
    High flow nasal canula & 1 per severe patient & $\ns$ & 15\\
    Nebuliser & 1 per severe patient & $\ns$ & 15\\
    Non-contact Thermometer & 1 for 20 out-patients & $\nq$ & 2\\
    Patient cot & 1 per severe and moderate & $\ns+\nin$ & 55 \\
    Wheel chair & 1 per 20 severe patients & $\ns/20$ & 1\\
    Stretcher & 1 per critical and 3 per 20 severe patients &
    $\nc + 3*\ns/20$ & 7\\
    Ambulance &  3 times total positive cases, each making 20 trips
    per day & $3 N / 20$ & 15\\
\bottomrule
\end{tabularx}
\end{table}
\end{landscape}



\begin{landscape}
\begin{table}
  \caption{Medical Consumables. The last column is
    rounded number assuming a total positive cases of $N=100$.}
  \begin{tabularx}{\linewidth}{XXcl}
    \toprule
    Item [Unit] & Reason & Formula & $N=100$ \\
    \midrule
Needles [per day]& 10 per critical, 5 per severe, 2 per moderate&
$10 \nc + 5 \ns + 2 \nin$ & 205 \\
Disposable Bags [per day] & 3 per critical 2 per severe and 1 per 4
moderate & $ 3\nc + 2\ns + 0.25\nin$ & 55 \\
Sanitizer [lt/day] & 250 ml per all in-patients & $0.25 (\nc+\ns+\nin)$
& 15 \\
Endotracheal Tube [per day] & 1 per critical patients for 3 days &
$\nc/3$ & 2 \\
Oxygen (medium) [cylinders per day] & 4 cylinders per critical patient
& $4 \nc$  & 20 \\ 
Central and Peripheral lines [per day] & 1 per critical patient for 3
days & $\nc/3$ & 2 \\
IV Fluids [lt/day] & 5 bottles of 500 ml per critical patient &
$2.5 \nc$ & 13 \\
Suction catheter [per day] & 1 per critical patient & $\nc$ & 5 \\
Test kits [per day] & 3 times the number of positive cases &
$3 N$ & 300  \\
\bottomrule
\end{tabularx}
\end{table}
\end{landscape}

\section{Acknowledgements}
The authors thank Dr Santosh Ansumali (JNCASR) and Dr. Aloke Kumar
(IISc) for providing the projection maps, and Dr. Soumyadeep
Bhattacharjee (Sankhyasutra Labs Pvt. Ltd) for data API. We also thank
the office of the Principal Scientific Advisor of Govt. of India for
providing an elaborate list of inventory from which some critical ones
have been used here.

\section{Errors and Omissions}
Though the authors have taken sufficient care in estimating the
quantity of the inventory, the proportions used here may change
depending on the actual field data in India for the patients and the
inventory usage pattern.  Please email your comments and suggestions
with justifications to \email{p.sunthar@iitb.ac.in} for inclusion in the
future updates of the software.


\bibliography{covid19lit}
\bibliographystyle{unsrtnat}

\end{document}
%%% Local Variables:
%%% mode: latex
%%% TeX-master: t
%%% End:
